\documentclass[10pt]{article}

\usepackage[margin=1in]{geometry}
\usepackage{amsmath}
\usepackage{amssymb}
\usepackage{amsthm}
\usepackage{mathtools}
\usepackage[shortlabels]{enumitem}
\usepackage[normalem]{ulem}
\usepackage{courier}

\usepackage{hyperref}
\hypersetup{
  colorlinks   = true, %Colours links instead of ugly boxes
  urlcolor     = black, %Colour for external hyperlinks
  linkcolor    = blue, %Colour of internal links
  citecolor    = blue  %Colour of citations
}

\usepackage[T1]{fontenc}
\usepackage{listings}
\lstset{
    language=HTML
    ,basicstyle=\ttfamily
    %,numbers=left
    ,breaklines=true
    }

%%%%%%%%%%%%%%%%%%%%%%%%%%%%%%%%%%%%%%%%%%%%%%%%%%%%%%%%%%%%%%%%%%%%%%%%%%%%%%%%

\theoremstyle{definition}
\newtheorem{problem}{Problem}
\newcommand{\E}{\mathbb E}
\newcommand{\R}{\mathbb R}
\DeclareMathOperator{\Var}{Var}
\DeclareMathOperator*{\argmin}{arg\,min}
\DeclareMathOperator*{\argmax}{arg\,max}

\newcommand{\trans}[1]{{#1}^{T}}
\newcommand{\loss}{\ell}
\newcommand{\w}{\mathbf w}
\newcommand{\mle}[1]{\hat{#1}_{\textit{mle}}}
\newcommand{\map}[1]{\hat{#1}_{\textit{map}}}
\newcommand{\normal}{\mathcal{N}}
\newcommand{\x}{\mathbf x}
\newcommand{\y}{\mathbf y}
\newcommand{\ltwo}[1]{\lVert {#1} \rVert}

%%%%%%%%%%%%%%%%%%%%%%%%%%%%%%%%%%%%%%%%%%%%%%%%%%%%%%%%%%%%%%%%%%%%%%%%%%%%%%%%

\begin{document}
\begin{center}
    {
\Large
    Week 03 Quiz
}

    \vspace{0.1in}
    CSCI040: \sout{Computing for the Web} Introduction to Hacking

    \vspace{0.1in}
\end{center}

\vspace{0.15in}
\noindent
\textbf{Total Score:} ~~~~~~~~~~~~~~~/10

\vspace{0.5in}
\noindent
\textbf{Printed Name:}

\noindent
\rule{\textwidth}{0.1pt}
\vspace{0.25in}

\noindent
\textbf{Collaboration Policy:}
\begin{enumerate}
    \item You MAY use any printed or handwritten notes.
    \item You MAY NOT use a computer or any other electronic device.
    \item You MAY NOT discuss this quiz with another human being who has not completed the quiz.
        This includes:
        \begin{enumerate}
            \item collaborating during the quiz, and
            \item telling a student in a different section the quiz was easy/hard.
        \end{enumerate}
\end{enumerate}

\vspace{0.15in}

\begin{problem}
    What is the output of the following code:
\end{problem}
\begin{lstlisting}
x = 2 * 2 
y = x // 4
z = y % 5
print("z=", z)
\end{lstlisting}
\vspace{1.5in}

\begin{problem}
    What is the output of the following code:
\end{problem}
\begin{lstlisting}
print(1 + 2 * 3 ** 4 % 5 // 6)
\end{lstlisting}
\vspace{1.5in}

\newpage
\begin{problem}
    What is the output of the following code:
\end{problem}
\begin{lstlisting}
x = 5%4
y = 5//4
if x == 1 and y == 1:
    result = 0
else:
    result = 1
print('result=', result)
\end{lstlisting}
\vspace{1.8in}


\begin{problem}
    What is the output of the following code:
\end{problem}
\begin{lstlisting}
if -3:
    result = 0
else:
    result = 1
print('result=', result)
\end{lstlisting}
\vspace{1.8in}


\begin{problem}
    What is the output of the following code:
\end{problem}
\begin{lstlisting}
i = 3
total = 0
while i < 5:
    total = total + i
    i += 1
print('total=', total)
\end{lstlisting}
\vspace{1.8in}


\begin{problem}
    What is the output of the following code:
\end{problem}
\begin{lstlisting}
total = 0
for i in range(3):
    total = total * 2
print('total=', total)
\end{lstlisting}
\vspace{2in}


\begin{problem}
    What is the output of the following code:
\end{problem}
\begin{lstlisting}
total = 0
for i in range(5, 0, -1):
    total += i
print("total=", total)
\end{lstlisting}
\vspace{2in}


\begin{problem}
    What is the output of the following code:
\end{problem}
\begin{lstlisting}
result = 1
for i in range(5):
    if i>3:
        result *= 2
    else:
        result *= (-1)
print('result=', result)
\end{lstlisting}
\vspace{2in}


\begin{problem}
    What is the output of the following code:
\end{problem}
\begin{lstlisting}
result = 1
for i in range(5):
    if i <= 3:
        result -= i
    else:
        result += 1
print('result=', result)
\end{lstlisting}
\vspace{2in}


\begin{problem}
    What is the output of the following code:
\end{problem}
\begin{lstlisting}
total = 0
for i in range(0, 10, 2):
    if i%2 == 2 and i<5:
        total += i
print("total=", total)
\end{lstlisting}
\vspace{2in}


\end{document}


