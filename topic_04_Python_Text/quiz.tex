\documentclass[10pt]{article}

\usepackage[margin=1in]{geometry}
\usepackage{amsmath}
\usepackage{amssymb}
\usepackage{amsthm}
\usepackage{mathtools}
\usepackage[shortlabels]{enumitem}
\usepackage[normalem]{ulem}

\usepackage{hyperref}
\hypersetup{
  colorlinks   = true, %Colours links instead of ugly boxes
  urlcolor     = black, %Colour for external hyperlinks
  linkcolor    = blue, %Colour of internal links
  citecolor    = blue  %Colour of citations
}

\usepackage{listings}
\lstset{language=Python} %,numbers=left}

%%%%%%%%%%%%%%%%%%%%%%%%%%%%%%%%%%%%%%%%%%%%%%%%%%%%%%%%%%%%%%%%%%%%%%%%%%%%%%%%

\theoremstyle{definition}
\newtheorem{problem}{Problem}
\newcommand{\E}{\mathbb E}
\newcommand{\R}{\mathbb R}
\DeclareMathOperator{\Var}{Var}
\DeclareMathOperator*{\argmin}{arg\,min}
\DeclareMathOperator*{\argmax}{arg\,max}

\newcommand{\trans}[1]{{#1}^{T}}
\newcommand{\loss}{\ell}
\newcommand{\w}{\mathbf w}
\newcommand{\mle}[1]{\hat{#1}_{\textit{mle}}}
\newcommand{\map}[1]{\hat{#1}_{\textit{map}}}
\newcommand{\normal}{\mathcal{N}}
\newcommand{\x}{\mathbf x}
\newcommand{\y}{\mathbf y}
\newcommand{\ltwo}[1]{\lVert {#1} \rVert}

%%%%%%%%%%%%%%%%%%%%%%%%%%%%%%%%%%%%%%%%%%%%%%%%%%%%%%%%%%%%%%%%%%%%%%%%%%%%%%%%

\begin{document}

\begin{center}
    {
\Large
Week 04 Quiz
}

    \vspace{0.1in}
    CSCI040: \sout{Computing for the Web} Introduction to Hacking

    \vspace{0.1in}
\end{center}


\vspace{0.15in}
\noindent
\textbf{Total Score:} ~~~~~~~~~~~~~~~/10

\vspace{0.5in}
\noindent
\textbf{Printed Name:}

\noindent
\rule{\textwidth}{0.1pt}
\vspace{0.25in}

\noindent
\textbf{Collaboration Policy:}
\begin{enumerate}
    \item You MAY use any printed or handwritten notes.
    \item You MAY NOT use a computer or any other electronic device.
    \item You MAY NOT discuss this quiz with another human being who has not completed the quiz.
        This includes:
        \begin{enumerate}
            \item collaborating during the quiz, and
            \item telling a student in a different section the quiz was easy/hard.
        \end{enumerate}
\end{enumerate}

\vspace{0.15in}

\begin{problem}
    What is the output of the following code:
\end{problem}
\begin{lstlisting}
total = 0
for i in range(3):
    total += i
for i in range(3):
    total += i
print('total=', total)
\end{lstlisting}
\vspace{1in}

\begin{problem}
    What is the output of the following code:
\end{problem}
\begin{lstlisting}
total = 0
for i in range(3):
    for j in range(3):
        total += 1
print('total=', total)
\end{lstlisting}
\vspace{1.5in}

\begin{problem}
    What is the output of the following code:
\end{problem}
\begin{lstlisting}
total = 0
for i in range(3):
    for j in range(i):
        total += i
print('total=', total)
\end{lstlisting}
\vspace{1.5in}

\begin{problem}
    What is the output of the following code:
\end{problem}
\begin{lstlisting}
total = 0
for i in range(1, 3):
    for j in range(0, 4, i):
        total += j
print('total=', total)
\end{lstlisting}
\vspace{1.5in}

\begin{problem}
    What is the output of the following code:
\end{problem}
\begin{lstlisting}
xs = [1, 3, 5, 7, 9, 11, 13, 15, 17, 19, 21]
total = 0
for i in range(2, 4):
    total += xs[i]
print('total=', total)
\end{lstlisting}

\newpage
\begin{problem}
    What is the output of the following code:
\end{problem}
\begin{lstlisting}
xs = [1, 3, 5]
total = 0
for x in xs:
    total += x
print('total=', total)
\end{lstlisting}
\vspace{1.5in}

\begin{problem}
    What is the output of the following code:
\end{problem}
\begin{lstlisting}
xs = [1, 3, 5]
total = 0
for i in range(len(xs)):
    total += xs[i]*i
print('total=', total)
\end{lstlisting}
\vspace{1.5in}


\begin{problem}
    What is the output of the following code:
\end{problem}
\begin{lstlisting}
xs = [1, 3, 5]
ys = [2, 4, 6]
total = 0
for y in ys:
    total += y
    for x in xs:
        total -= x*y
print('total=', total)
\end{lstlisting}
\vspace{1.5in}


\newpage
\begin{problem}
    What is the output of the following code:
\end{problem}
\begin{lstlisting}
xss = [[2, 4, 6], [1, 3], [0, 1, 2, 3]]
total = 0
for xs in xss:
    total += xs[-1]
    for x in xs:
        total -= x
print('total=', total)
\end{lstlisting}
\vspace{1.5in}


\begin{problem}
    What is the output of the following code:
\end{problem}
\begin{lstlisting}
xss = [[2, 4, 6], [1, 3], [0, 1, 2, 3]]
total = 0
for i in range(len(xss)):
    for j in range(len(xss[i])):
        total += xss[-i-1][-j]
print('total=', total)
\end{lstlisting}
\vspace{1.5in}
\end{document}
