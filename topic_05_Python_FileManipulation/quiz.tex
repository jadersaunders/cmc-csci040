\documentclass[10pt]{article}

\usepackage[margin=1in]{geometry}
\usepackage{amsmath}
\usepackage{amssymb}
\usepackage{amsthm}
\usepackage{mathtools}
\usepackage[shortlabels]{enumitem}
\usepackage[normalem]{ulem}

\usepackage{hyperref}
\hypersetup{
  colorlinks   = true, %Colours links instead of ugly boxes
  urlcolor     = black, %Colour for external hyperlinks
  linkcolor    = blue, %Colour of internal links
  citecolor    = blue  %Colour of citations
}

\usepackage{listings}
\lstset{language=Python} %,numbers=left}

%%%%%%%%%%%%%%%%%%%%%%%%%%%%%%%%%%%%%%%%%%%%%%%%%%%%%%%%%%%%%%%%%%%%%%%%%%%%%%%%

\theoremstyle{definition}
\newtheorem{problem}{Problem}
\newcommand{\E}{\mathbb E}
\newcommand{\R}{\mathbb R}
\DeclareMathOperator{\Var}{Var}
\DeclareMathOperator*{\argmin}{arg\,min}
\DeclareMathOperator*{\argmax}{arg\,max}

\newcommand{\trans}[1]{{#1}^{T}}
\newcommand{\loss}{\ell}
\newcommand{\w}{\mathbf w}
\newcommand{\mle}[1]{\hat{#1}_{\textit{mle}}}
\newcommand{\map}[1]{\hat{#1}_{\textit{map}}}
\newcommand{\normal}{\mathcal{N}}
\newcommand{\x}{\mathbf x}
\newcommand{\y}{\mathbf y}
\newcommand{\ltwo}[1]{\lVert {#1} \rVert}

%%%%%%%%%%%%%%%%%%%%%%%%%%%%%%%%%%%%%%%%%%%%%%%%%%%%%%%%%%%%%%%%%%%%%%%%%%%%%%%%

\begin{document}

\begin{center}
    {
\Large
    Week 05 Quiz
}

    \vspace{0.1in}
    CSCI040: \sout{Computing for the Web} Introduction to Hacking

    \vspace{0.1in}
\end{center}


\vspace{0.15in}
\noindent
\textbf{Total Score:} ~~~~~~~~~~~~~~~/10

\vspace{0.5in}
\noindent
\textbf{Printed Name:}

\noindent
\rule{\textwidth}{0.1pt}
\vspace{0.25in}

\noindent
\textbf{Collaboration Policy:}
\begin{enumerate}
    \item You MAY use any printed or handwritten notes.
    \item You MAY NOT use a computer or any other electronic device.
    \item You MAY NOT discuss this quiz with another human being who has not completed the quiz.
        This includes:
        \begin{enumerate}
            \item collaborating during the quiz, and
            \item telling a student in a different section the quiz was easy/hard.
        \end{enumerate}
\end{enumerate}

\vspace{0.15in}

\begin{problem}
    What is the output of the following code:
\end{problem}
\begin{lstlisting}
s = 'hello world'
print(s[:4])
\end{lstlisting}
\vspace{1in}

\begin{problem}
    What is the output of the following code:
\end{problem}
\begin{lstlisting}
s = '12345'
xs = ["1", "2", "3", "4", "5"]
b = s[1] == xs[1]
print('b=', b)
\end{lstlisting}
\vspace{1.5in}

\newpage
\begin{problem}
    What is the output of the following code:
\end{problem}
\begin{lstlisting}
s = 'python is awesome'
s.replace('python', 'everything')
print('s=', s)
\end{lstlisting}
\vspace{1.5in}

\begin{problem}
    What is the output of the following code:
\end{problem}
\begin{lstlisting}
s = 'python  is  awesome'
s.split()
t = s[-1]
print('t=', t)
\end{lstlisting}
\vspace{1.5in}

\begin{problem}
    What is the output of the following code:
\end{problem}
\begin{lstlisting}
s = 'python is awesome'
t = s.split()
' '.join(t)
print('t=', t)
\end{lstlisting}
\vspace{1.5in}

\newpage
\begin{problem}
    What is the output of the following code:
\end{problem}
\begin{lstlisting}
s = 'Guido von Rossum'
i = s.find(' ')
t = s[i+1:]
print('t=',t)
\end{lstlisting}
\vspace{1.5in}

\begin{problem}
    What is the output of the following code:
\end{problem}
\begin{lstlisting}
b = ',' < 'a'
print('b=', b)
\end{lstlisting}
\vspace{1.5in}


\begin{problem}
    What is the output of the following code:
\end{problem}
\begin{lstlisting}
s = 'aaA BBb ABAB'
xs = s.split()
xs.sort()
print('xs=', xs)
\end{lstlisting}
\vspace{1.5in}


\newpage
\begin{problem}
    What is the output of the following code:
\end{problem}
\begin{lstlisting}
print(0x10)
\end{lstlisting}
\vspace{1.5in}


\begin{problem}
    What is the output of the following code:
\end{problem}
\begin{lstlisting}
s = '\x4c\x4f\x4c'
print('s=', s)
\end{lstlisting}
\vspace{1.5in}
\end{document}
