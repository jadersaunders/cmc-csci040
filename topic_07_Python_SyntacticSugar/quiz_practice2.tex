\documentclass[10pt]{article}

\usepackage[margin=1in]{geometry}
\usepackage{amsmath}
\usepackage{amssymb}
\usepackage{amsthm}
\usepackage{mathtools}
\usepackage[shortlabels]{enumitem}
\usepackage[normalem]{ulem}

\usepackage{hyperref}
\hypersetup{
  colorlinks   = true, %Colours links instead of ugly boxes
  urlcolor     = black, %Colour for external hyperlinks
  linkcolor    = blue, %Colour of internal links
  citecolor    = blue  %Colour of citations
}

\usepackage{listings}
\lstset{language=Python,breaklines=true} %,numbers=left}

%%%%%%%%%%%%%%%%%%%%%%%%%%%%%%%%%%%%%%%%%%%%%%%%%%%%%%%%%%%%%%%%%%%%%%%%%%%%%%%%

\theoremstyle{definition}
\newtheorem{problem}{Problem}
\newcommand{\E}{\mathbb E}
\newcommand{\R}{\mathbb R}
\DeclareMathOperator{\Var}{Var}
\DeclareMathOperator*{\argmin}{arg\,min}
\DeclareMathOperator*{\argmax}{arg\,max}

\newcommand{\trans}[1]{{#1}^{T}}
\newcommand{\loss}{\ell}
\newcommand{\w}{\mathbf w}
\newcommand{\mle}[1]{\hat{#1}_{\textit{mle}}}
\newcommand{\map}[1]{\hat{#1}_{\textit{map}}}
\newcommand{\normal}{\mathcal{N}}
\newcommand{\x}{\mathbf x}
\newcommand{\y}{\mathbf y}
\newcommand{\ltwo}[1]{\lVert {#1} \rVert}

%%%%%%%%%%%%%%%%%%%%%%%%%%%%%%%%%%%%%%%%%%%%%%%%%%%%%%%%%%%%%%%%%%%%%%%%%%%%%%%%

\begin{document}

\begin{center}
    {
\Large
    Week 08 Quiz Question Bank
}

    \vspace{0.1in}
    CSCI040: \sout{Computing for the Web} Introduction to Hacking

    \vspace{0.1in}
\end{center}


\noindent
\rule{\textwidth}{0.1pt}
\vspace{0.01in}

\noindent
\textbf{NOTE:}
The quiz will be 5 questions long, similar to the practice quiz.
The questions on the quiz will be similar to the practice quiz OR this question bank.

\vspace{0.1in}
\noindent
\rule{\textwidth}{0.1pt}
\vspace{0.3in}

\begin{problem}
    What is the output of the following code:
\end{problem}
\begin{lstlisting}
names = ['alice', 'bob', 'charlie', 'dave', 'eve']
accumulator = []
for i,name in enumerate(names):
    text = 'number ' + str(i) + ' is ' + name
    accumulator.append(text)
print('accumulator[-1]=', accumulator[-1])
\end{lstlisting}
\vspace{1.5in}

\begin{problem}
    What is the output of the following code:
\end{problem}
\begin{lstlisting}
sentence = 'python is *weird*'
accumulator = ''
for i,char in enumerate(sentence):
    if char == '*' and sentence[i-1] == ' ':
        accumulator += '_'
    elif char == '*':
        accumulator += '_!'
    else:
        accumulator += char
print('accumulator=', accumulator)
\end{lstlisting}
\vspace{1.5in}

\newpage
\begin{problem}
    What is the output of the following code:
\end{problem}
\begin{lstlisting}
x = 1 if 2 > 3 else 4
print('x=', x)
\end{lstlisting}
\vspace{1.5in}

\begin{problem}
    What is the output of the following code:
\end{problem}
\begin{lstlisting}
x = 'if' if '<' > 'else' else 'if'
print('x=', x)
\end{lstlisting}
\vspace{1.5in}

\begin{problem}
    What is the output of the following code:
\end{problem}
\begin{lstlisting}
names = ['alice', 'bob', 'charlie', 'dave', 'eve']
greetings = [ name[0].upper() + name[1:].lower() for name in names ]
greeting = greetings[1]
print('greeting=', greeting)
\end{lstlisting}
\vspace{1.5in}

\begin{problem}
    What is the output of the following code:
\end{problem}
\begin{lstlisting}
xs = [ x**x for x in range(-3,3) ]
num = xs[5]
print('num=', num)
\end{lstlisting}
\vspace{1.5in}

\begin{problem}
    What is the output of the following code:
\end{problem}
\begin{lstlisting}
xs = [ x for x in range(-3,3) if x ]
num = xs[3]
print('num=', num)
\end{lstlisting}
\vspace{1.2in}


\begin{problem}
    What is the output of the following code:
\end{problem}
\begin{lstlisting}
sentence = "This is an example sentence with a few words in it."
words = [ word.lower() for word in sentence.split() if 't' in word ]
print('len(words)=', len(words))
\end{lstlisting}
\vspace{1.2in}


\begin{problem}
    What is the output of the following code:
\end{problem}
\begin{lstlisting}
sentence = "This is an example sentence with a few words in it."
xs = [ x.upper() for x in sentence if 't' in x ]
print('xs[1]=', xs[1])
\end{lstlisting}
\vspace{1.2in}


\begin{problem}
    What is the output of the following code:
\end{problem}
\begin{lstlisting}
sentence = "This is an example sentence with a few words in it."
small_words = [ word.lower() for word in sentence.split() if len(word) <= 2]
print('len(small_words)=', len(small_words))
\end{lstlisting}
\vspace{1.2in}


\newpage
\begin{problem}
    What is the output of the following code:
\end{problem}
\begin{lstlisting}
xss = [ [ i for i in range(x) ] for x in [2, 3, 4] ]
x = xss[-1][-2]
print('x=', x)
\end{lstlisting}
\vspace{3in}

\begin{problem}
    What is the output of the following code:
\end{problem}
\begin{lstlisting}
xss = [ [ i for i in range(x) ] for x in [2, 3, 4]  if x%2 == 1 ]
x = xss[-1][-2]
print('x=', x)
\end{lstlisting}
\vspace{2.0in}


\newpage
\begin{problem}
    What is the output of the following code:
\end{problem}
\begin{lstlisting}
xss = [ [ i for i in range(x) if i%3== 1 ] for x in [4, 5, 6]  if x%2 == 1 ]
x = xss[-1][-1]
print('x=', x)
\end{lstlisting}
\vspace{3.0in}


\begin{problem}
    What is the output of the following code:
\end{problem}
\begin{lstlisting}
xss = [ [ i for i in range(x) if x%3== 1 ] for x in [4, 5, 6]  if x%2 == 1 ]
print('xss=', xss)
\end{lstlisting}
\vspace{2.0in}



\newpage
\begin{problem}
    What is the output of the following code:
\end{problem}
\begin{lstlisting}
tweets = [
    { "source": "Twitter Web Client"
    , "text": "From Donald Trump: Wishing everyone a wonderful holiday & a happy, healthy, prosperous New Year. Let\u2019s think like champions in 2010!"
    , "retweet_count": 28
    }, 
    { "source": "Twitter Web Client"
    , "text": "Trump International Tower in Chicago ranked 6th tallest building in world by Council on Tall Buildings & Urban Habitat http://bit.ly/sqvQq"
    , "retweet_count": 33
    },
    { "source": "Twitter Web Client"
    , "text": "Wishing you and yours a very Happy and Bountiful Thanksgiving!"
    , "retweet_count": 13
    },
    { "source": "Twitter for iPhone"
    , "text": "RT @realDonaldTrump: Happy Birthday @DonaldJTrumpJr!\nhttps://t.co/uRxyCD3hBz"
    , "retweet_count": 9529
    },
    { "source": "Twitter for iPhone"
    , "text": "Happy Birthday @DonaldJTrumpJr!\nhttps://t.co/uRxyCD3hBz"
    , "retweet_count": 9529
    },
    { "source": "Twitter for Android"
    , "text": "Happy New Year to all, including to my many enemies and those who have fought me and lost so badly they just don't know what to do. Love!"
    , "retweet_count": 141853
    },
    { "source": "Twitter for Android"
    , "text": "Russians are playing @CNN and @NBCNews for such fools - funny to watch, they don't have a clue! @FoxNews totally gets it!"
    , "retweet_count": 23213
    },
    { "source": "Twitter for iPhone"
    , "text": "Join @AmerIcan32, founded by Hall of Fame legend @JimBrownNFL32 on 1/19/2017 in Washington, D.C.\u2026 https://t.co/9WJZ8iTCQV"
    , "retweet_count": 7366
    }]

popular_tweets = [tweet for tweet in tweets if tweet['retweet_count'] > 100]
print('len(popular_tweets)=', len(popular_tweets))
\end{lstlisting}
\vspace{1.5in}
\end{document}

