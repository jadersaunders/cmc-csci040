\documentclass[10pt]{article}

\usepackage[margin=1in]{geometry}
\usepackage{amsmath}
\usepackage{amssymb}
\usepackage{amsthm}
\usepackage{mathtools}
\usepackage[shortlabels]{enumitem}
\usepackage[normalem]{ulem}

\usepackage{hyperref}
\hypersetup{
  colorlinks   = true, %Colours links instead of ugly boxes
  urlcolor     = black, %Colour for external hyperlinks
  linkcolor    = blue, %Colour of internal links
  citecolor    = blue  %Colour of citations
}

\usepackage{listings}
\lstset{language=Python,breaklines=true} %,numbers=left}

%%%%%%%%%%%%%%%%%%%%%%%%%%%%%%%%%%%%%%%%%%%%%%%%%%%%%%%%%%%%%%%%%%%%%%%%%%%%%%%%

\theoremstyle{definition}
\newtheorem{problem}{Problem}
\newcommand{\E}{\mathbb E}
\newcommand{\R}{\mathbb R}
\DeclareMathOperator{\Var}{Var}
\DeclareMathOperator*{\argmin}{arg\,min}
\DeclareMathOperator*{\argmax}{arg\,max}

\newcommand{\trans}[1]{{#1}^{T}}
\newcommand{\loss}{\ell}
\newcommand{\w}{\mathbf w}
\newcommand{\mle}[1]{\hat{#1}_{\textit{mle}}}
\newcommand{\map}[1]{\hat{#1}_{\textit{map}}}
\newcommand{\normal}{\mathcal{N}}
\newcommand{\x}{\mathbf x}
\newcommand{\y}{\mathbf y}
\newcommand{\ltwo}[1]{\lVert {#1} \rVert}

%%%%%%%%%%%%%%%%%%%%%%%%%%%%%%%%%%%%%%%%%%%%%%%%%%%%%%%%%%%%%%%%%%%%%%%%%%%%%%%%

\begin{document}

\begin{center}
    {
\Large
    Exceptions Quiz
}

    \vspace{0.1in}
    CSCI040: \sout{Computing for the Web} Introduction to Hacking

    \vspace{0.1in}
\end{center}


\vspace{0.15in}
\noindent
\textbf{Total Score:} ~~~~~~~~~~~~~~~/10

\vspace{0.5in}
\noindent
\textbf{Printed Name:}

\noindent
\rule{\textwidth}{0.1pt}
\vspace{0.25in}

\noindent
\textbf{Collaboration Policy:}
\begin{enumerate}
    \item You MAY use any printed or handwritten notes.
    \item You MAY NOT use a computer or any other electronic device.
    \item You MAY NOT discuss this quiz with another human being who has not completed the quiz.
        This includes:
        \begin{enumerate}
            \item collaborating during the quiz, and
            \item telling a student in a different section the quiz was easy/hard.
        \end{enumerate}
\end{enumerate}

%\noindent
%\textbf{NOTE:}
%On the real quiz, there will be at least 2 of the ``harder'' problems from the practice question bank.
\vspace{0.15in}


\begin{problem}
    The following code (circle one)

    \vspace{0.25in}
    \hspace{0.5in}terminates without error 
    \hspace{1in}throws an exception
    \hspace{1in}runs forever
    \vspace{0.25in}

    \noindent
    If the code terminates without error, write the output.
    If the code throws an exception, state the exception.
\end{problem}
\begin{lstlisting}
for x in [0, 1]:
    if x:
        total += 1
print('total=', total)
\end{lstlisting}
\vspace{1in}

\newpage
\begin{problem}
    The following code (circle one)

    \vspace{0.25in}
    \hspace{0.5in}terminates without error 
    \hspace{1in}throws an exception
    \hspace{1in}runs forever
    \vspace{0.25in}

    \noindent
    If the code terminates without error, write the output.
    If the code throws an exception, state the exception.
\end{problem}
\begin{lstlisting}
xs = [1, 2, 3, ]
while xs:
    xs.append('test')
    assert('t' in xs)
print('len(xs)=',len(xs))
\end{lstlisting}
\vspace{2in}


\begin{problem}
    The following code (circle one)

    \vspace{0.25in}
    \hspace{0.5in}terminates without error 
    \hspace{1in}throws an exception
    \hspace{1in}runs forever
    \vspace{0.25in}

    \noindent
    If the code terminates without error, write the output.
    If the code throws an exception, state the exception.
\end{problem}
\begin{lstlisting}
xs = [ {'hw1':99,'hw2':88}, {'hw1':82,'hw2':91} ]
alice = 0
bob = 1
try:
    output = grades['alice']['hw2']
except IndexError:
    output += 'oops'
print('output=', output)
\end{lstlisting}
\vspace{1in}

\newpage
\begin{problem}
    The following code (circle one)

    \vspace{0.25in}
    \hspace{0.5in}terminates without error 
    \hspace{1in}throws an exception
    \hspace{1in}runs forever
    \vspace{0.25in}

    \noindent
    If the code terminates without error, write the output.
    If the code throws an exception, state the exception.
\end{problem}
\begin{lstlisting}
grades={
    'alice':{'hw1':99,'hw2':88},
    'bob':{'hw1':82,'hw2':91},
}
try:
    output = "grade=" + grades['bob']['hw1']
except KeyError:
    output = 'oops'
print('output=', output)
\end{lstlisting}
\vspace{2in}



\begin{problem}
    The following code (circle one)

    \vspace{0.25in}
    \hspace{0.5in}terminates without error 
    \hspace{1in}throws an exception
    \hspace{1in}runs forever
    \vspace{0.25in}

    \noindent
    If the code terminates without error, write the output.
    If the code throws an exception, state the exception.
\end{problem}
\begin{lstlisting}
grades={
    'alice':{'hw1':99,'hw2':88},
    'bob':{'hw1':82,'hw2':91},
}
try:
    output = "grade="
    output += grades['alice']['hw3']
except IndexError:
    output += 'oops'
print('output=', output)
\end{lstlisting}
\vspace{1.5in}
\end{document}
